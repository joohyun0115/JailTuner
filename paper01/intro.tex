\section{Introduction} \label{sec:introduction}
%$$$$$$$$$$$$$$$$$$$$$$$$$$$$$$$$$$$$$$$$$$$$$$$$$$$$$$$$$$$$$$$$$$$$$$$$$$$$$$$$
%$$$$$$$$$$$$$$$$$$$$$$$$$$$$$$$$$$$$$$$$$$$$$$$$$$$$$$$$$$$$$$$$$$$$$$$$$$$$$$$$
%Background
%$$$$$$$$$$$$$$$$$$$$$$$$$$$$$$$$$$$$$$$$$$$$$$$$$$$$$$$$$$$$$$$$$$$$$$$$$$$$$$$$
\ifkor
HPC를 위한 스파크가 필요하다. 
리눅스 운영체제가 많이 사용된다. 
기존 연구들은 HPC에 spark를 적용하기 위한 방법들이 연구되고 있다.
15코어 이하에서 확장성을 검사하였음.
하지만, 이러한 연구들은 리눅스 운영체제의 특징을 활용하지 못하는 문제점이 있다.
리눅스 운영체제 확장성에 대해서 가장 중요한것은 
리눅스는 conflit free한 운영체제가 아니다.
이유는~!!! 여러가지 상황을 고려하였기 때문에 
만약 응용프로그램이 scalalabe하게 디자인이 되어 있다면, 리눅스라도 스케일러블한 
상황으로 만들 수 있다[]. 
\else

\fi

%$$$$$$$$$$$$$$$$$$$$$$$$$$$$$$$$$$$$$$$$$$$$$$$$$$$$$$$$$$$$$$$$$$$$$$$$$$$$$$$$
%$$$$$$$$$$$$$$$$$$$$$$$$$$$$$$$$$$$$$$$$$$$$$$$$$$$$$$$$$$$$$$$$$$$$$$$$$$$$$$$$
%Problem
%$$$$$$$$$$$$$$$$$$$$$$$$$$$$$$$$$$$$$$$$$$$$$$$$$$$$$$$$$$$$$$$$$$$$$$$$$$$$$$$$
\ifkor
스케일러블 한 Scale-Up 서버를 위한 연구들이 진행되고 있지만, 




\else


\fi



%$$$$$$$$$$$$$$$$$$$$$$$$$$$$$$$$$$$$$$$$$$$$$$$$$$$$$$$$$$$$$$$$$$$$$$$$$$$$$$$$
%$$$$$$$$$$$$$$$$$$$$$$$$$$$$$$$$$$$$$$$$$$$$$$$$$$$$$$$$$$$$$$$$$$$$$$$$$$$$$$$$
%본 연구에 해결책
%$$$$$$$$$$$$$$$$$$$$$$$$$$$$$$$$$$$$$$$$$$$$$$$$$$$$$$$$$$$$$$$$$$$$$$$$$$$$$$$$
%$$$$$$$$$$$$$$$$$$$$$$$$$$$$$$$$$$$$$$$$$$$$$$$$$$$$$$$$$$$$$$$$$$$$$$$$$$$$$$$$
\ifkor
Scalability 문제를 해결하기 위해, 본 연구는 메모리 파티션 기반의 JailDocker를 만들었다. 
JailDocker는 Scale-Up server에서 최적의 성능을 내기 위한 파티션 방법이다. 
기반을 사용하기 우리는 Docker를 사용하였다. 

\else


\fi




%$$$$$$$$$$$$$$$$$$$$$$$$$$$$$$$$$$$$$$$$$$$$$$$$$$$$$$$$$$$$$$$$$$$$$$$$$$$$$$$$
%본 연구에서 기여한 것
%$$$$$$$$$$$$$$$$$$$$$$$$$$$$$$$$$$$$$$$$$$$$$$$$$$$$$$$$$$$$$$$$$$$$$$$$$$$$$$$$
\ifkor


\else

\fi


