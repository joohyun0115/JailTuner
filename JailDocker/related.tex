\section{Related work} \label{sec:RelatedWork}

%$$$$$$$$$$$$$$$$$$$$$$$$$$$$$$$$$$$$$$$$$$$$$$$$$$$$$$$$$$$$$$$$$$$$$$$$$$$$$$$$
%Paragraph 1:Linux Scalability의 연구에 대한 설명
%$$$$$$$$$$$$$$$$$$$$$$$$$$$$$$$$$$$$$$$$$$$$$$$$$$$$$$$$$$$$$$$$$$$$$$$$$$$$$$$$
\ifkor
\noindent
\textbf{Apache Spark Scalability.}
To improve the scalability, researchers have attempted to create new
operating systems~\cite{Boyd-WickizerCorey}~\cite{Wentzlaff2010fOS}
%~\cite{Baumann2009Barrelfish}~\cite{Zellweger2014Multikernel}
%~\cite{Liu2009Tessellation}~\cite{Farrington2010Helios}
or have
attempted to optimize existing operating systems~\cite{SilasBoydWickizer2010LinuxScales48}~\cite{AustinTClements2012RCUBalancedTrees}~\cite{Clements2013RadixVM}~\cite{SilasBoydWickizerPth}
%~\cite{Changwoo2016UMSF}.
Our research belongs to optimizing existing operating systems in order to
solve the Linux fork scalability problem.
However, previous research did not deal with the anonymous reverse mapping,
which is one of the fork scalability bottleneck.

\else

\fi

%$$$$$$$$$$$$$$$$$$$$$$$$$$$$$$$$$$$$$$$$$$$$$$$$$$$$$$$$$$$$$$$$$$$$$$$$$$$$$$$$
%Paragraph 1:Manycore Scalability의 연구에 대한 설명
%$$$$$$$$$$$$$$$$$$$$$$$$$$$$$$$$$$$$$$$$$$$$$$$$$$$$$$$$$$$$$$$$$$$$$$$$$$$$$$$$
\ifkor
\noindent
\textbf{Manycore Scale-up Server Scalability.}
To improve the scalability, researchers have attempted to create new
operating systems~\cite{Boyd-WickizerCorey}~\cite{Wentzlaff2010fOS}
%~\cite{Baumann2009Barrelfish}~\cite{Zellweger2014Multikernel}
%~\cite{Liu2009Tessellation}~\cite{Farrington2010Helios}
or have
attempted to optimize existing operating systems~\cite{SilasBoydWickizer2010LinuxScales48}~\cite{AustinTClements2012RCUBalancedTrees}~\cite{Clements2013RadixVM}~\cite{SilasBoydWickizerPth}
%~\cite{Changwoo2016UMSF}.
Our research belongs to optimizing existing operating systems in order to
solve the Linux fork scalability problem.
However, previous research did not deal with the anonymous reverse mapping,
which is one of the fork scalability bottleneck.

\else

\fi
