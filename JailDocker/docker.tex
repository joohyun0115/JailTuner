\section{Scalable Partitioning}
%$$$$$$$$$$$$$$$$$$$$$$$$$$$$$$$$$$$$$$$$$$$$$$$$$$$$$$$$$$$$$$$$$$$$$$$$$$$$$$$$
%$$$$$$$$$$$$$$$$$$$$$$$$$$$$$$$$$$$$$$$$$$$$$$$$$$$$$$$$$$$$$$$$$$$$$$$$$$$$$$$$
% 일반적인 매니코어 또는 Scale-server의 scalability 대한 설명과 이번장에 대한 설명
%$$$$$$$$$$$$$$$$$$$$$$$$$$$$$$$$$$$$$$$$$$$$$$$$$$$$$$$$$$$$$$$$$$$$$$$$$$$$$$$$
\ifkor
파티션닝 방법이 필요한 이유는 spark library와 runtime엔진이 single node에 동작하는 
시스템의 scalability 특성을 고려하지 않았기 때문이다. 
scale-up server를 위한 spark scalability의 근본적인 해결 방법은 spark library와 
runtime엔진을 scale-up서버를 위해 scalable하게 만드는 것이다.
하지만 scale-out 시스템의 scalability를 위해 작성된 spark의 library와 runtime 엔진을 
수정하는것은 쉽지않다.
이러한 single node로 구성된 manycore scale-up 서버에 대한 scalability
문제는 도커를 활용한 파티셔닝 기법을 사용하여 해결할 수 있다.
이번 장에서는 우리가 수행한 파티션닝 방법이 필요한 이유와 우리가 수행한 방법에 대해서 설명한다.
\else

\fi

%$$$$$$$$$$$$$$$$$$$$$$$$$$$$$$$$$$$$$$$$$$$$$$$$$$$$$$$$$$$$$$$$$$$$$$$$$$$$$$$$
%$$$$$$$$$$$$$$$$$$$$$$$$$$$$$$$$$$$$$$$$$$$$$$$$$$$$$$$$$$$$$$$$$$$$$$$$$$$$$$$$
% NUMA 영향에 대한 설명
%$$$$$$$$$$$$$$$$$$$$$$$$$$$$$$$$$$$$$$$$$$$$$$$$$$$$$$$$$$$$$$$$$$$$$$$$$$$$$$$$
\ifkor
파티션닝 방법이 필요한 가장 큰 이유는 DRAM access latency 때문이다. 
만약 scale-up server가 NUMA 아키텍쳐를 가진 경우일 경우, 
리눅스는 이러한 문제를 해결하기 위해 커널 내부에 automatic NUMA balancing이라는 기능이 있으나, 
아직 파티션되어 수행하는 방법보다는 성능이 떨어진다[]. 그림 xx-xx는 NUMA balancing을 
사용한 방법과 파티션기법을 사용한 방법의 성능 측정한 결과를 보여준다. 
\else

\fi

%$$$$$$$$$$$$$$$$$$$$$$$$$$$$$$$$$$$$$$$$$$$$$$$$$$$$$$$$$$$$$$$$$$$$$$$$$$$$$$$$
%$$$$$$$$$$$$$$$$$$$$$$$$$$$$$$$$$$$$$$$$$$$$$$$$$$$$$$$$$$$$$$$$$$$$$$$$$$$$$$$$
% Linux kernel scalability (lock, cache cohearnci, scheduler)등등 OS 노이즈에 대한 설명
%$$$$$$$$$$$$$$$$$$$$$$$$$$$$$$$$$$$$$$$$$$$$$$$$$$$$$$$$$$$$$$$$$$$$$$$$$$$$$$$$
\ifkor
NUMA의 영향 뿐만 아니라, operating system의 scalability 저해 요소 때문에 파티션닝 방법은 필요하다.
Shared memory 시스템의 공유데이터 때문에 발생하는 scalability 저해 요소 때문에 필요하다.
첫째로 공유 데이터를 lock이 있다. 표 xxx 앞에서 실험한 spark의 wordcount에 대해서 .
JVM 위에서 동작하는 thread간의 공유하는 single address space때문에 발생하는 공유 문제이다.
다음으로 scheduler가 아직 
마지막으로 cache cohearci traffic이 있다. 

\else

\fi

%$$$$$$$$$$$$$$$$$$$$$$$$$$$$$$$$$$$$$$$$$$$$$$$$$$$$$$$$$$$$$$$$$$$$$$$$$$$$$$$$
%$$$$$$$$$$$$$$$$$$$$$$$$$$$$$$$$$$$$$$$$$$$$$$$$$$$$$$$$$$$$$$$$$$$$$$$$$$$$$$$$
% 스파크는 결국 : shared memory system -> distributed system 처럼해야한다. 
%$$$$$$$$$$$$$$$$$$$$$$$$$$$$$$$$$$$$$$$$$$$$$$$$$$$$$$$$$$$$$$$$$$$$$$$$$$$$$$$$
\ifkor
이처럼 NUMA와 shared memory의 공유 데이터 때문에 발생하는 scalability 저해 요소 때문에, 
scale-up 서버를 위한 스파크도 distributed system의 개념처럼 동작해야한다.
따라서 본 연구는 메모리와 CPU를 파티션닝을 하여 마치 shared memory 시스템을 distributed system 
처럼 동작하도록 제안 한다.
스마크 워커들은 모두 독립적인 cpu와 memory를 할당받아 최대한 thread간의 공유메모리와 remote
memory에 접근을 막도록 하였다.
\else

\fi




%$$$$$$$$$$$$$$$$$$$$$$$$$$$$$$$$$$$$$$$$$$$$$$$$$$$$$$$$$$$$$$$$$$$$$$$$$$$$$$$$
%$$$$$$$$$$$$$$$$$$$$$$$$$$$$$$$$$$$$$$$$$$$$$$$$$$$$$$$$$$$$$$$$$$$$$$$$$$$$$$$$
% 제안하는 구조 framework 설명
%$$$$$$$$$$$$$$$$$$$$$$$$$$$$$$$$$$$$$$$$$$$$$$$$$$$$$$$$$$$$$$$$$$$$$$$$$$$$$$$$
\ifkor
본 연구에서 제안하는 구조는 그림 <x>와 같다. 

가장 최적의 방법은 best fit을 찾는 과정이 필요하다. 


\else

\fi




%$$$$$$$$$$$$$$$$$$$$$$$$$$$$$$$$$$$$$$$$$$$$$$$$$$$$$$$$$$$$$$$$$$$$$$$$$$$$$$$$
%$$$$$$$$$$$$$$$$$$$$$$$$$$$$$$$$$$$$$$$$$$$$$$$$$$$$$$$$$$$$$$$$$$$$$$$$$$$$$$$$
% Docker를 이용하는 이유
%$$$$$$$$$$$$$$$$$$$$$$$$$$$$$$$$$$$$$$$$$$$$$$$$$$$$$$$$$$$$$$$$$$$$$$$$$$$$$$$$
\ifkor
우리는 Worker들의 파티션닝을 위해 Docker container를 사용하였다. 
그 이유는 virtual machine보다 훨씬 가벼운 구조로 되어 있으며,  
최근 docker기반으로 시스템을 관리하는 구조로 변경되고 있기 때문이다.
따라서 본 연구의 파티셔닝을 위한 방법으로 Docker container를 사용하였다.
\else

\fi




